\documentclass[12pt,a4paper]{article}

\title{Alternatives for Neighborhood Function in Kohonen Maps}

\author{Iliyan Zankinski, Kolyu Kolev, Todor Balabanov}

% Iliyan Zankinski iliyan@hsi.iccs.bas.bg
% Kolyu Kolev kkolev@iit.bas.bg 
% Todor Balabanov todorb@iinf.bas.bg

% \affil{Institute of Information and Communication Technologies,
% Bulgarian Academy of Sciences,
% acad. G. Bonchev Str, Block 2, 1113 Sofia, Bulgaria,
% todorb@iinf.bas.bg}

\date{\empty}

\begin{document} 
\maketitle

In the field of the artificial intelligence artificial neural networks are one of the most researched topics. Multilayer perceptron has a reputation for the most used type of artificial neural network, but other other types, as Kohonen maps, are also very interesting. Proposed by Teuvo Kohonen in the 1980s, self-organizing maps have application in meteorology, oceanography, project prioritization and selection, seismic facies analysis for oil and gas exploration, failure mode and effects analysis, creation of artwork and many other areas. Self-organizing maps are very useful for visualization by data dimensions reduction. Unsupervised competitive learning is used in self-organizing maps and the basic idea is the net to classify input data in predefined number of clusters. When the net is with smaller number of nodes it achieve results similar to K-means clustering. One of the components in the self-organizing maps is the neighborhood function. It gives the distance between the neuron one neuron and other neuron in particular step. The simplest form of a neighborhood function is 1 for the closest nodes and 0 for all other, but the most used neighborhood function is a Gaussian function. In this research fading cosine and exponential regulated cosine functions are proposed as alternatives for neighborhood function.

\end{document}